%!TEX root = ../paper.tex

\section{Text Preprocessing}

The review dataset provides the raw text scraped from IMDB, meaning we still have to extract our features from the text.
In Figure~\ref{fig:preprocessing} we present our preprocessing steps.
The input is a list of raw text files.
The output is also a list of raw text files, however, it has been modified by each step in the process.
Next we will go more indepth on each of the preprocessing steps that were applied on the raw text.

\begin{figure*}[ht!]
\setlength{\unitlength}{0.14in}
\centering
\begin{picture}(33,10)
      % picture environment with the size (dimensions)
      % 32 length units wide, and 15 units high.
    \put(2,4){\framebox(5,3){\footnotesize{Emoji}}}
    \put(8,4){\framebox(5,3){\footnotesize{Stop Words}}}
    \put(14,4){\framebox(5,3){\footnotesize{Lemmatizing}}}
    \put(20,4){\framebox(5,3){\footnotesize{Negation}}}
    \put(26,4){\framebox(5,3){\footnotesize{Clean}}}
    \put(1,1.5){\framebox(31,8){}}
    
    \put(0,5.5){\vector(1,0){2}}
    \put(7,5.5){\vector(1,0){1}}
    \put(13,5.5){\vector(1,0){1}}
    \put(19,5.5){\vector(1,0){1}}
    \put(25,5.5){\vector(1,0){1}}
    \put(31,5.5){\vector(1,0){2}}
    
\end{picture}
\caption{Pipeline for the Preprocessing}
\label{fig:preprocessing}
\end{figure*}
    

\mysubsection{Emoji Tagging}
Many reviews are written very informally, meaning different kinds of emoji's are used that express different sentiments.
Because the goal is to determine the sentiment of each review, emoji's can express valuable information that should be maintained.
Using the vocabular list and corresponding sentiment score given with the dataset, we were able to extract which emoji's are linked with positive or negative sentiment.
Each of those Emoji's was finally replaced in the text with \textit{happy\_smiley} or \textit{sad\_smiley}, making it a possible feature in our dataset.

\mysubsection{Remove Stop Words}
Natural language contains words that are not adding any value when it is used to train a model.
Those words are referred to as stopwords because they are the most common words in the language.
To improve performance, we can remove all stop words. After all they are the most common words in the language and not per se tied to a negative or positive class.

\mysubsection{Lemmatizing and Stemming}
Similar meaning words can occur in multiple ways (\eg, \textit{am} and \textit{is} are different forms of the same word \textit{be}) even though they should be combined together, because they are exchangeable.
There are two common approaches for this: (1) Stemming and (2) Lemmatizing. 
We chose to use \textit{Lemmatizing}, because it does a morphological analysis, which will still change the words, but the actual spelling would still stay correct.
An example from Manning \etal \cite{Manning:2008} shows that a stemmer would replace the following words \textit{operate operating operates} to \textit{oper}, while a lemmatizer would give the correct option \textit{operate}.

\mysubsection{Negation Handling}
In natural language we often make use of negatives and sometimes even double negatives. For example, a sententce like: this movie was not good.
Could still be classified as positive, because \textit{good} is linked with a positive sentiment. However, the \textit{not} flips the sentiment.
By using negation handling, we take that into account by adding \textit{\_NEG} to each word that has a \textit{not} in front of it.
Similarly, if we encounter a double negation we stop adding a \textit{\_NEG} because the original sentiment of the words is left intact.

\mysubsection{Clean Text}
Finally, to finish the preprocessing the text needs to be cleaned up from punctuation, or other odd symbols that don't carry any meaning in natural language.
